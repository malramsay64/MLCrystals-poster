% vim:ft=tex:
%
\documentclass{beamer}

\makeatletter
\def\input@path{{usyd-beamer-theme/}}
\makeatother

\usepackage[orientation=portrait,size=a0,scale=1.5,debug]{beamerposter}
\usepackage[utf8]{inputenc}
\usepackage[australian]{babel}
\usepackage{amsmath}
\usepackage{subcaption}

\usepackage[backend=biber,style=phys]{biblatex}
\usepackage{biblatex}
\addbibresource{references.bib}
\setbeamertemplate{bibliography item}{\insertbiblabel}
\renewcommand*{\bibfont}{\footnotesize}

\mode<presentation>{\usetheme{usyd-poster}}

\setbeamertemplate{footline}{
  \begin{beamercolorbox}[wd=\paperwidth]{upper separation line foot}
    \rule{0pt}{3pt}
  \end{beamercolorbox}

  \leavevmode%
  \begin{beamercolorbox}[ht=10ex,leftskip=1em,rightskip=1em]{author in head/foot}%
    \hfill
    \normalsize\texttt{mram5995@uni.sydney.edu.au}\hspace{2em}
    \vskip1ex
  \end{beamercolorbox}
  \vskip0pt%
}


\graphicspath{{figures/}{usyd-beamer-theme/}}

\title{The Detection and Characterisation of \\\vspace{0.3em} Molecular Crystals}
\author{Malcolm Ramsay and Peter Harrowell}
\institute{School of Chemistry, The University of Sydney}

\begin{document}
\begin{frame}[t]{}
  \begin{columns}[t]

    \begin{column}{0.47\linewidth}

      \begin{block}{Abstract}
        The wide range of crystal structures of a molecular crystal
        can make it difficult to detect and characterise each of them individually.
        Presented, is a general approach to the detection and characterisation
        of local crystalline order using machine learning.
        This approach is able to categorise different crystalline orderings and the liquid phase
        with an accuracy in excess of 95\% over a wide temperature range.
      \end{block}

      \begin{block}{Introduction}
        The molecule we are studying has three distinct crystal structures
        with potential energies all within 2\% of each other.
        \begin{figure}[h]
          \centering
          \begin{subfigure}[t]{0.3\linewidth}
              \includegraphics[width=\linewidth]{trimer-crys-pg}
            \caption{pg}
            \label{fig:crystals pg}
          \end{subfigure}
          \begin{subfigure}[t]{0.3\linewidth}
              \includegraphics[width=\linewidth]{trimer-crys-p2}
            \caption{p2}
            \label{fig:crystals p2}
          \end{subfigure}
          \begin{subfigure}[t]{0.3\linewidth}
              \includegraphics[width=\linewidth]{trimer-crys-p2gg}
            \caption{p2gg}
            \label{fig:crystals p2gg}
          \end{subfigure}
          \caption{The three lowest energy structures of the Trimer molecule.}
          \label{fig:crystals}
        \end{figure}
        A complete understanding of the liquid-solid phase transition requires
        the detection and characterisation of all
        these crystal structures as they transition to and from the liquid phase.
        The diversity of structure required metrics and parameters tailored for each structure.

      \end{block}

    \begin{block}{Machine Learning Methodology}

      There has recently been work using machine learning to
      categorize fcc and bcc crystal structures~\autocite{Reinhart2017,Dietz2017} for single atom
      potentials.
      No comparable work could be found for molecular crystals,
      so the relative orientation of neighbouring molecules was chosen as the feature
      to distinguish the crystal structures over a range of temperatures.

      \begin{figure}[h]
        \centering
        \includegraphics[width=0.67\linewidth]{orientations}
        \caption{The features of a molecule (in red) are the relative orientations of the nearest
        neighbours (grey).}
        \label{fig:orientations}
      \end{figure}

      Machine learning algorithms from the scikit-learn~\autocite{scikit-learn} library
      were used for the classification. The K-nearest neighbours algorithm was the most
      suitable for this work, being both highly accuracy and fast to run.

    \end{block}

    \begin{block}{References}
      \printbibliography
      \vspace{0.5em}
    \end{block}

    \end{column}
    \begin{column}{0.47\linewidth}
      \begin{block}{Results}
        A single classification algorithm can be used to monitor melting of all crystals
        with minimal classification errors or noise from thermal fluctuations.

        \begin{figure}[h]
          \centering
          \begin{subfigure}[t]{\linewidth}
            \centering
              \includegraphics[width=\linewidth]{trimer-cat-pg}
            \caption{pg}
            \label{fig:categorised pg}
          \end{subfigure}
          \begin{subfigure}[t]{\linewidth}
            \centering
              \includegraphics[width=\linewidth]{trimer-cat-p2}
            \caption{p2}
            \label{fig:categorised p2}
          \end{subfigure}
          \begin{subfigure}[t]{\linewidth}
            \centering
              \includegraphics[width=\linewidth]{trimer-cat-p2gg}
            \caption{p2gg}
            \label{fig:categorised p2gg}
          \end{subfigure}
          \caption{Each of the crystal structures characterised using the same
            machine learning algorithm. Regions classified as crystalline are darker
            than those classifies as liquid.}
          \label{fig:categorised}
        \end{figure}

        The most notable feature of these crystal structures is that the p2gg structure
        has undergone a solid-state phase transition to the p2 structure, with the p2gg
        structure at the grain boundary of the two orientations. This algorithm is able
        to identify this transition taking place, correctly classifying each polymorph.

      \end{block}

      \begin{block}{Future Work}

        This work could be extend to a wide range of crystal structures using the
        relative orientations, a scaled distance, and angle to neighbours. The extension
        of this technique requires configurations containing the liquid and the crystal
        structures of a set of molecules which can be used to train the model.

        Further work would also incorporate the machine learning functionality into a
        domain specific, simple to use open source library, enabling this analysis
        to become a standard tool.

        \vspace{0.62em}

      \end{block}

    \end{column}
  \end{columns}
\end{frame}

\end{document}
